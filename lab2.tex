\documentclass[a4paper,12pt]{article}

% ==============================
%            ПАКЕТЫ
% ==============================
\usepackage{cmap}
\usepackage[T2A]{fontenc}
\usepackage[utf8]{inputenc}
\usepackage[english,russian]{babel}
\usepackage{hyperref}
\usepackage{mathtools}
\usepackage{amssymb}
\usepackage{enumitem}
\usepackage{tikz}
\usetikzlibrary{automata, positioning, arrows.meta}
\usepackage[linesnumbered,ruled,vlined]{algorithm2e}



% ==============================
%         СОКРАЩЕНИЯ
% ==============================
\newcommand{\e}{\varepsilon}
\newcommand{\T}{\tau}
\newcommand{\s}{^\ast}



% ==============================
%         ПЕРВЫЙ ЛИСТ
% ==============================

\author{Лиганкина Анна\\ Вариант 20}
\title{Лабораторная работа №2}
\date{\today}

\begin{document}

\maketitle
\tableofcontents

\newpage



% ==============================
%           ЗАДАНИЕ
% ==============================

\section{Задание}

Дано академическое регулярное выражение:
\[
    (aa | bb | cc)\s b (aaa | bbb)\s ((ab | bc | ccc)\s aa)\s abc(a | b | c)(b | \e)
\]
По нему необходимо построить:
\begin{itemize}
    \item минимальный ДКА, распознающий его язык (минимальность обосновать таблицей 
    классов эквивалентности);
    
    \item возможно малый НКА, распознающий его язык. Возможно малый переключающийся 
    (с конъюнкцией) КА, распознающий его язык. Частично обосновать таблицами множеств 
    классов эквивалентности;
    
    \item расширенное регулярное выражение, распознающее тот же язык. В расширенном 
    выражении можно использовать:
    \begin{itemize}
        \item wildcard-операцию . для замены произвольного символа алфавита;
        
        \item положительную итерацию $\T^+$ и опцию $\T?$ 
        \[\T^+ = \T\T\s, \T? = (\T|\e));\]
        
        \item операции предпросмотра 
        $\T_0(?=\T_1)\T_2 \equiv \T_0((\T_1.\s) \cap \T_2)$ 
        и ретроспективной проверки 
        $\T_0(? <= \T_1)\T_2 \equiv (\T_0 \cap (\T_1.\s))\T_2$, 
        а также их отрицательные версии 
        $\T_0(?! \; \T_1)\T_2 \equiv \T_0(\overline{(\T_1.\s)} \cap \T_2)$ и 
        $\T_0(?<! \; \T_1)\T_2 \equiv (\T_0 \cap \overline{(\T_1.\s)})\T_2$;
        
        \item классы букв $[c_1 . . . c_k] \equiv (c_1|c_2| . . . |c_k)$ и их 
        дополнения $[ˆc_1 . . . c_k]$;
        
        \item (обязательно) маркеры начала и конца выражения ˆ и \$.
    \end{itemize}
\end{itemize}

Провести автоматическое тестирование предполагаемой эквивалентности построенных 
распознавателей. Тем самым необходимо построить алгоритмы, определяющие принадлежность 
слова языку академического регулярного выражения, ДКА, НКА и ПКА.
% -------------------------------------------------------------------------------------
\section{Построение автоматов}



% ==============================
%              НКА
% ==============================
\subsection{НКА}





% ==============================
%              ДКА
% ==============================
\subsection{ДКА}





% ==============================
%             ПКА
% ==============================
\subsection{ПКА}





% ==============================
%           EXTRAREG
% ==============================
\subsection{Расширенное регулярное выражение}





% -------------------------------------------------------------------------------------
\newpage
\section{Тестирование}





\end{document}