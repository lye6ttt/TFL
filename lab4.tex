\documentclass[a4paper,12pt]{article}

% ==============================
%            ПАКЕТЫ
% ==============================
\usepackage{cmap}
\usepackage[T2A]{fontenc}
\usepackage[utf8]{inputenc}
\usepackage[english,russian]{babel}
\usepackage{hyperref}
\usepackage{mathtools}
\usepackage{amssymb}

% ==============================
%         ПЕРВЫЙ ЛИСТ
% ==============================

\author{Лиганкина Анна\\ Вариант 20}
\title{Лабораторная работа №4}
\date{\today}

\begin{document}

\maketitle
\tableofcontents

\newpage

% ==============================
%           ЗАДАНИЕ
% ==============================

\section{Задание}

Дан язык:
\[
    L = \textasciicircum ((?\!\!: a|b)^*)c((?\!\!: a|b)^*)
    (?\!\!= \backslash 1 ab \backslash 2)\backslash 2 ba \backslash 1 \$
\]
Необходимо:
\begin{itemize}
    \item [--] проанализировать язык на КС-свойство, в случае его наличия - на 
    регулярность;

    \item [--] построить <<наивный>> парсер слов для языка, используя рекурсивный 
    разбор с возвратами (парсер не должен зацикливаться);

    \item [--] построить оптимизированный парсер слов для языка. Оценить сверху
    его вычислительную сложность;

    \item [--] посредством фазз-тестирования проверить эквивалентность парсеров
    и построить сравнительные графики их времени работы на случайных словах,
    принадлежащих языку и не принадлежащих языку (два тестовых пула).
\end{itemize}


\section{Исследование языка}

Заметим, что исходный язык можно переписать в виде:
\[
    L =\{ w \;|\; w = xcyybax \;\&\; 
    x = (a|b)^* \;\&\; y = (a|b)^* \;\&\; xaby = ybax \}
\]
так как длина предпросмотра совпадает с длиной конца слова ($|\backslash 1 ab \backslash 2| = |\backslash 2 ba \backslash 1|$).

\subsection*{Доказательство, что язык не является КС}

Так как контекстно-свободные языки замкнуты относительно пересечения с регулярными
языками, то пересечем $L$ с $R = a^*ca^*ba^+$ и докажем, что получившийся язык
$L_R = L \cap R$ не КС.
\\\\
Слово $x$ однозначно определяется разделителем $c$ и состоит из букв $a$, либо
является пустым словом. Пусть $x = a^n, n \geq 0$. Так как в словах из $R$
содержится лишь одна буква $b$, то она соответствует букве $b$ из обязательного
блока $ba$, а значит, $y$ тоже состоит из букв $a$. Пусть $y = a^m, m \geq 0$.
\\\\
Рассмотрим условие $xaby = ybax$:
\[
    a^naba^m = a^mbaa^n
\]\[
    a^{n+1}ba^m = a^mba^{n+1} \Rightarrow m = n + 1
\]
Таким образом, язык $L_R$ выглядит так:
\[
    L = \{ w \;|\; w = a^nca^{2n+2} ba^{n+1} \;\&\; n \geq 0 \}
\]
По лемме о накачке (длина накачки $N$) рассмотрим слово $a^Nca^{2N+2}ba^{N+1}$ 
$\in L$:
\begin{itemize}
    \item если $|vxy| < N$ попадает в один из блоков из букв $a$, то нарушается 
    баланс с другими буквами $a$, и мы выходим из языка;

    \item если $|vxy| < N$ попадает на стык блоков букв $a$ и накачка задевает
    букву $c$ или букву $b$, то нарушается структура слова, где лишь одна буква
    $b$ и одна буква $c$, и мы выходим из языка;

    \item если $|vxy| < N$ попадает на стык блоков букв $a$ и мы качаем одновременно
    2 блока букв $a$ (3 не можем, так как $|vxy| < N$), то все равно рушится баланс
    с оставшимся блоком $a$, и мы выходим из языка.
\end{itemize}
Поэтому язык $L_R$ не является контекстно-свободным, а, следовательно, и язык $L$.

\end{document}