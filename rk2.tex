\documentclass[a4paper,12pt]{article}
\usepackage{geometry}
\geometry{a4paper,tmargin=2cm,bmargin=2cm,lmargin=2cm,rmargin=2cm}
\sloppy

% ==============================
%            ПАКЕТЫ
% ==============================
\usepackage{cmap}
\usepackage[T2A]{fontenc}
\usepackage[utf8]{inputenc}
\usepackage[english,russian]{babel}
\usepackage{tikz}
\usetikzlibrary{automata, positioning, arrows.meta}
\usepackage{amsmath}

% ==============================
%         ПЕРВЫЙ ЛИСТ
% ==============================

\author{Лиганкина Анна\\ Вариант 20}
\title{Рубежный контроль №2}
\date{\today}

\begin{document}

\maketitle

% ==============================
%          ЗАДАНИЕ 1
% ==============================

\section*{Задание 1}

Дан язык SRS $bc \to cb, ac \to ccc$ над множеством базисных слов $a^nb^{n+k}c^k$. 
Определить, является ли язык КС.

\subsection*{Решение}

Инварианты данной SRS:
\begin{enumerate}
    \item количество букв $b$ не меняется, т.е. остается $n+k$;
    \item буквы $c$ могут двигаться относительно букв $b$ только влево;
    \item количество букв $a$ уменьшается, причем число $2|w|_a + |w|_c = 2n+k$ постоянно (количество
    $c$ может только увеличиваться на 2 за счет одной буквы $a$).
\end{enumerate}
Поэтому данный язык можно представить как:
\begin{enumerate}
    \item В случае, если правило $ac \to ccc$ еще не было применено:
    \[
        L_1 = \{ w \;\Big|\; w = a^n z \;\&\; z = (b|c)^* \;\&\; |z|_b = n+k \;\&\; |z|_c = k \}
    \]
    \item Если правило $ac \to ccc$ было применено хотя бы раз:
    \[
        L_2 = \{ w \;\Big|\; w = a^{n-i}c^{2i+1}z \;\&\; z = (b|c)^* \;\&\; 
        |z|_b = n+k \;\&\; |z|_c = k-1 \;\&\; 
    \]\[
        i, k \geq 1 \;\&\; n \geq i \}
    \]
\end{enumerate}
Таким образом, исходный язык SRS $L = L_1 \cup L_2$. Поскольку КС языки замкнуты относительно
объединения, для доказательства того, что $L$ - КС, достаточно показать, что $L_1$ и $L_2$ - КС.

\subsubsection*{Язык $L_1$}

\[
    L_1 = \{ w \;\Big|\; w = a^n z \;\&\; z = (b|c)^* \;\&\; |z|_b = n+k \;\&\; |z|_c = k \}
\]

Для $L_1$ можно построить недетеминированный PDA $\Rightarrow L_1$ - КС: 
\begin{center}
\begin{tikzpicture}[
    ->, >=Stealth, node distance=3cm, 
    every state/.style={thick, fill=gray!10, minimum size=1cm},
    initial text={},
    every edge/.style={draw, thick}
    ]
    
    \node[state, initial] (0) {};
    \node[state, right of=0] (1) {1};
    \node[state, right of=1] (2) {2};
    \node[state, accepting, right of=2] (3) {3};
    \node[state, below=2.5cm of 3] (4) {4};
    
    \node[left=1.5cm of 0] (start) {};
    
    \path (0) edge [bend left=20] node[above, sloped] {$a, z_0/Az_0$} (1);
    \path (0) edge [bend right=40] node[below, sloped] {$c, z_0/Az_0$} (2);
    \path (0) edge [bend right=30] node[below, sloped] {$b, z_0/Az_0$} (4);
    
    \path (1) edge [loop above] node[above] {$a, x/Ax$} (1);
    \path (1) edge [bend left=15] node[above, sloped] {$b, A/\varepsilon$} (2);
    \path (1) edge [bend right=5] node[below, sloped, pos=0.4] {$c, x/Ax$} (2);
    
    \path (2) edge [loop above] node[above] {$b, A/\varepsilon$} (2);
    \path (2) edge [loop below] node[below] {$c, x/Ax$} (2);
    \path (2) edge [bend left=15] node[above, sloped] {$\varepsilon, z_0/z_0$} (3);
    
    \path (3) edge [bend left=5] node[below, sloped, pos=0.4] {$c, z_0/Az_0$} (2);
    \path (3) edge [bend right=15] node[left, yshift=-20pt] {$b, z_0/Az_0$} (4);
    
    \path (4) edge [loop right] node[right] {$c, A/\varepsilon$} (4);
    \path (4) edge [loop below] node[below] {$b, x/Ax$} (4);
    \path (4) edge [bend right=15] node[right] {$\varepsilon, z_0/z_0$} (3);
    
    \draw[->, thick] (start) -- (0);
\end{tikzpicture}
\end{center}

\subsubsection*{Язык $L_2$}

Для удобства во втором случае переобозначим $n = n - i, k = k - 1$. Тогда
\[
    L_2 = \{ w \;\Big|\; w = a^n c^{2i+1}z \;\&\; z = (b|c)^* \;\&\; |z|_b = n+k+i+1 \;\&\; |z|_c = k
\]\[
    \;\&\; i > 0 \;\&\; n, k \geq 0 \}
\]
Для $L_2$ можно построить недетеминированный PDA $\Rightarrow L_2$ - КС: 
\begin{center}
\begin{tikzpicture}[
    ->, >=Stealth, node distance=3cm, 
    every state/.style={thick, fill=gray!10, minimum size=1cm},
    initial text={},
    every edge/.style={draw, thick}
    ]
    
    \node[state, initial] (0) {};
    \node[state, above=1cm of 0] (1) {1};
    \node[state, right=1cm of 1] (2) {2};
    \node[state, right=1.5cm of 2] (3) {3};
    \node[state, right=1.5cm of 3] (4) {4};
    \node[state, right=1.5cm of 4] (5) {5};
    \node[state, below=1.5cm of 5] (6) {6};
    \node[state, accepting, left=2.5cm of 6] (7) {7};
    \node[state, left=2.5cm of 7] (8) {8};
    
    \node[left=1.5cm of 0] (start) {};
    
    \path (0) edge [bend left=15] node[left] {$a, z_0/Az_0$} (1);
    \path (0) edge [bend right=0] node[left] {$c$} (2);
    
    \path (1) edge [loop above] node[above] {$a, x/Ax$} (1);
    \path (1) edge [bend left=15] node[above] {$c$} (2);
    
    \path (2) edge [bend left=15] node[above, sloped] {$c, x/Ax$} (3);
    
    \path (3) edge [bend right=15] node[below] {$c$} (4);
    
    \path (4) edge [bend right=15] node[above] {$c, x/Ax$} (3);
    \path (4) edge [bend left=15] node[above, sloped] {$c, x/Ax$} (5);
    \path (4) edge [bend right=0] node[right, xshift=-15pt, yshift=20pt] {$b$} (6);
    
    \path (5) edge [bend left=70] node[right] {$b$} (6);

    \path (6) edge [loop above] node[above] {$c, x/Ax$} (1);
    \path (6) edge [loop below] node[right, xshift=5pt, yshift=10pt] {$b, A/\varepsilon$} (6);

    \path (6) edge [bend right=15] node[above] {$\varepsilon, z_0/z_0$} (7);
    \path (7) edge [bend right=15] node[below] {$c, z_0/Az_0$} (6);

    \path (8) edge [bend left=15] node[above] {$\varepsilon, z_0/z_0$} (7);
    \path (7) edge [bend left=15] node[below] {$b, z_0/Az_0$} (8);
    \path (8) edge [loop above] node[above] {$b, x/Ax$} (8);
    \path (8) edge [loop below] node[left, xshift=-7pt, yshift=10pt] {$c, A/\varepsilon$} (8);
    
    \draw[->, thick] (start) -- (0);
\end{tikzpicture}
\end{center}
Таким образом, язык $L_1$ и $L_2$ - КС, а значит, $L$ - КС. При этом для определения принадлежности
слова языку обязателен счетчик $\Rightarrow$ язык нерегулярный.\\
\textbf{Ответ}: контекстно-свободный, нерегулярный.


% ==============================
%          ЗАДАНИЕ 2
% ==============================

\section*{Задание 2}

Дан язык $\{ w_1w_2 \;\Big|\; |w_1| > 1 \;\&\; w_2 = z_1w_1^Rz_2 \;\&\; |z_1| \neq |z_2| \}$.
Определить, является ли он DCFL. Если да, то исследовать на LL-свойства.

\subsection*{Решение}

Поскольку длина $|w|$ не фиксирована, и т.к. если $w^R$ содержится в слове, то для любой подстроки $w$
существует как подстрока слова $w^R$ ее реверс, можно ограничиться рассмотрением в качестве $w$ только
первых двух букв. Однако возникают следующие ситуации:
\begin{enumerate}
    \item $\underbrace{ab}_{w}aa\underbrace{ba}_{w^R}ba$, хотя у слова есть разбиение 
    $\underbrace{ab}_{w}aaba\underbrace{ba}_{w^R}$. Поэтому, если в слове несколько раз встречается 
    реверс первых двух букв (причем с перекрытиями), то слово автоматически принадлежит языку (т.к.
    хотя бы одно из разбиений будет подходить критерию $|z_1| \neq |z_2|$).

    \item $\underbrace{ab}_{w}a^n\underbrace{ba}_{w}a^n, n \geq 2$ - $ba$ встречается лишь один раз, но
    существует разбиение \\
    $\underbrace{aba}_{w}a^{n-2}\underbrace{aba}_{w}a^n \Rightarrow w \in L$. Таким образом, еще одна 
    необходимая проверка в слове - наличие реверса первых трех букв. Если это условие выполнено, то 
    слово автоматически принадлежит языку.
\end{enumerate}

Поскольку DCFL замкнуты относительно дополнения, то построим язык $L'$ слов, не принадлежащих $L$.
Нам подходят слова из алфавита $\{a, b\}^*$ такие, что в остатке слова либо не встречается реверс
двух первых букв, либо встречается один раз, причем предшествующая им буква не совпадает с третьей,
и длины отсекаемых слов равны.
Половина DPDA - в файле DPDA.dot/DPDA.svg. Остальная часть - зеркальная (аналогично)

\begin{figure}[!htb]
    \centering
    \includegraphics[width=\linewidth]{dpda.png}
\end{figure}

Таким образом, язык $L$ - DCFL. Исследуем теперь язык на LL-свойства.
\\

Язык $L$ не является LL, что доказывается по методу неожиданного конца:
\\
Возьмем префикс длины $> N$: $x = aa b^{N+k} aa$ ($N$ такое, что после чтения $N$ букв $b$ в стеке не 
меньше $k+3$ стековых символов). Также языку принадлежит слово с тем же префиксом $x$ и суффиксом 
$b^{N+k+1}$. За чтение $b^{k}aa$ израсходуется не более $k + 2$ стековых символа, а значит, для
слова с префиксом $x$ и суффиксом $\varepsilon$ стек окажется непустым $\Rightarrow$ язык не LL(k).
\\
\textbf{Ответ}: DCFL, не LL(k).


% ==============================
%          ЗАДАНИЕ 3
% ==============================
\newpage

\section*{Задание 3}

Дана атрибутная грамматика:\\
\[
\begin{array}{rclll}
S & \to & TbS         & ; & S_0.a := S_1.a + 1, \; S_1.a == T.a \\
S & \to & T           & ; & S.a := T.a \\
T & \to & aTa         & ; & T_0.a := T_1.a + 1 \\
T & \to & bT          & ; & T_0.a := T_1.a \cdot 2 \\
T & \to & bbT         & ; & T.a := 0 \\
T & \to & aTa         & ; & T_0.a := T_1.a + 2 \\
T & \to & \varepsilon & ; & T.a := 0
\end{array}
\]
Определить, является ли язык КС.

\subsection*{Решение}

Докажем, что пересечение языка $L$, порожденного данной атрибутной грамматикой, с регулярным языком
\[
    R = aaaa \; (aa)^* b^+ \; aa
\]
или по-другому
\[
    R = \{ a^{2p} b^q aa, p \geq 2, q \geq 1 \}
\]
не является контекстно-свободным, а значит, и $L$ не является КС.\\
$L_R = L \cap R$ имеет вид:
\[
\begin{array}{rclll}
S & \to & TbT         & ; & T_1.a == T_2.a \\
T & \to & aTa         & ; & T_0.a := T_1.a + 1 \\
T & \to & bT          & ; & T_0.a := T_1.a \cdot 2 \\
T & \to & bbT         & ; & T.a := 0 \\
T & \to & aTa         & ; & T_0.a := T_1.a + 2 \\
T & \to & \varepsilon & ; & T.a := 0
\end{array}
\]
Обоснование. (От противного). Пусть $w \in L_R$ порождено правилом $S \to T$. Тогда, т.к. в начале стоит
минимум 4 буквы $a$, то применяется правило $T \to aTa$ минимум 4 раза, что невозможно, поскольку
в конце стоит только 2 буквы $a$ подряд. Поэтому имеем четкое разбиение $w$ по правилу $S \to TbS$, где 
$T = aaaa \; a^*$, а $S = b^+aa$. 
\\\\
При этом последняя $S$ может быть раскрыта только по правилу $S \to T$, потому что если предположить, 
что  $S \to TbS$, то $T$ содержит только $b^*$, а значит, атрибут $a$ для этого $T$ равен 0. Поэтому 
(при любом представлении $S = b^* aa$) вес этого слова должен быть равен 0. Таким образом, $w$ не
принадлежит языку, поскольку вес левой части $T$ точно больше 0.
\\\\
По этой же причине нужно рассматривать блок $b^*$ в качестве применения правила $T \to bT$, которое
умножает атрибут $a$ (иначе опять уйдем в 0). Получается, что значения атрибута $a$ оценивается так
(с учетом правила $S \to T_1bT_2$):
\begin{itemize}
    \item для $T_1:$ $2 \leq p \leq T_1.a \leq 2p$;
    \item для $T_2:$ $2 \leq 2^{q - 1} \leq T_2.a \leq 2^{q - 1} \cdot 2$
\end{itemize}
Таким образом, $T_1.a = T_2.a \Leftrightarrow$
\[
\begin{cases}
    2^{q - 1} \leq 2p        & p \geq 2^{q - 2} \\
    p \leq 2^{q - 1} \cdot 2 & p \leq 2^q
\end{cases}
\Leftrightarrow 2^ {q - 1} \leq 2p \leq 2^{q + 1}
\]
Т.е. $L_R = \{a^{2p}b^qaa \;\Big|\; p \geq 2 \;\&\;  2^ {q - 1} \leq 2p \leq 2^{q + 1} \}$. Теперь
воспользуемся леммой о накачке:
\\
Возьмем слово $w = a^{2^{q-1}}b^qaa$, где $q > N$ - длина накачки. Рассмотрим разбиение $w = uvxyz$,
где $|vxy| < N$:
\begin{itemize}
    \item $vxy \subset a^{2p}$: при отрицательной накачке ($k = 0$) выходим из языка, т.к. получаем 
    слово $a^{2p - t}b^qaa$, где $2p = 2^{q-1}, t > 0$;

    \item $vxy \subset b^q$: при любой положительной накачке ($k > 0$) выходим из языка, т.к. нарушается
    условие $2^{q-1} = 2p \geq 2^{q+t-1}$;

    \item $vxy$ попадает на стык $a^{2p}$ и $b^q$: число накачиваемых букв $a$ меньше $N$, а значит
    новое число букв $a < N \cdot k$, а в части $b^q$ накачивается хотя бы 1 буква, поэтому новое число
    букв $b \geq q + 1$. Получается, новое число букв $a$
    \[
        2^{q-1} + N \cdot k < 2^{q-1} + q \cdot k \;\text{ т.к.}\; (q > N) < 2^{q-1} + 2^{q-1} \cdot k 
        \leq 2^q
    \]
    и мы выходим из языка;

    \item $vxy$ не может задеть блок $aa$, потому что это автоматически выводит из языка.
\end{itemize}
Таким образом, согласно лемме о накачке, язык $L_R$ не является КС, а значит, и исходный язык не является КС.
\\
\textbf{Ответ}: не контекстно-свободный.
\end{document}